\section{Ausblick}

Zum Zeitpunkt der Abgabe dieses Projektes, lieferte der genetischen Algorithmus schon relativ gute Ergebnisse. Trotzdem k�nnte der genetische Algorithmus noch erweitert werden, um bessere Ergebnisse zu liefern. Es w�re denkbar weitere Mutationen hinzuzuf�gen, beispielsweise eine Mutation zum Drehen von Teilen einer Tour. Um die Ergebnisse des genetischen Algorithmus zu verbessern, k�nnten auch noch unterschiedliche Kombinationen der Parameter getestet werden, beispielsweise die Verteilung der bereits vorhandenen Mutationen.Ebenfalls eine M�glichkeit bessere Ergebnisse zu erzielen, ist die Implementierung eines weiteren Crossovers bzw. Verbesserung des vorhandenen.

Die Applikation bietet dem Anwender mit verschiedenen Diagrammen Informationen �ber den Ablauf des genetischen Algorithmus, z.B. Fitness und L�nge eines Plans in Abh�ngigkeit der Iterationen oder Verteilung der Fitnesswerte. Allerdings k�nnten auch noch weitere Diagramme bzw. Informationen hinzugef�gt werden, z.B. Anzahl der Fahrzeuge, Wartezeiten oder Gesamtfahrzeit.

In diesem Zusammenhang w�re es f�r den Anwender vorteilhaft, wenn er zur Laufzeit des genetischen Algorithmus �nderungen an den Parametern vornehmen k�nnte. Eine M�glichkeit w�re das dynamische �ndern der Werte f�r die Wichtigkeit der Fitnessbausteine.

Momentan wird die Applikation noch aus der Entwicklungsumgebung Eclipse gestartet und die �nderung der Parameter muss direkt im Sourcecode erfolgen. Die �nderung der Parameter erfolgt zwar an einer zentralen Stelle, jedoch muss man Kenntnis �ber den Aufbau des Programms besitzen. Hier w�re es w�nschenswert, wenn die Parameter von au�en in die Applikation gegeben werden k�nnen. Bestenfalls sollte dies mit einer grafischen Eingabemaske realisiert werden.

Au�erdem sind Verbesserungen an der Softwarearchitektur denkbar, beispielsweise k�nnte noch die Dependency-Injection eingef�hrt werden, um Abh�ngigkeiten zwischen Objekten zentral verwalten zu k�nnen.

Es wurden bis jetzt nur f�r einige Module Unittests geschrieben. Es w�re wichtig, dass fehlende Unittests noch erg�nzt werden, um nicht offensichtliche Fehler zu finden und im Anschluss zu beheben.

Eine weitere �berlegung w�re, die Werte (L�nge eines Plans, Anzahl Fahrzeuge, Fitnesswert) jeder Iteration in eine Datei auszugeben, um im Anschluss detaillierte Auswertungen durchf�hren zu k�nnen.

Momentan beinhaltet die Applikation noch nicht die Umsetzung der Synchronisation zwischen zwei Auftr�gen mit verschiedenen Produkten. Diese Funktionalit�t m�sste in der weiteren Entwicklung noch hinzugef�gt werden.
