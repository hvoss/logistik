\section{Ausblick}
Zum Zeitpunkt der Abgabe dieses Projektes, lieferte der genetischen Algorithmus schon relativ gute Ergebnisse. Trotzdem könnte der genetische Algorithmus noch erweitert werden, um bessere Ergebnisse zu liefern. Es wäre denkbar weitere Mutationen hinzuzufügen, beispielsweise eine Mutation zum Drehen von Teilen einer Tour. Um die Ergebnisse des genetischen Algorithmus zu verbessern, könnten auch noch unterschiedliche Kombinationen der Parameter getestet werden, beispielsweise die Verteilung der bereits vorhandenen Mutationen. Ebenfalls eine Möglichkeit bessere Ergebnisse zu erzielen, ist die Implementierung eines weiteren Crossovers bzw. Verbesserung des vorhandenen.

Die Applikation bietet dem Anwender mit verschiedenen Diagrammen Informationen über den Ablauf des genetischen Algorithmus, z.B. Fitness und Länge eines Plans in Abhängigkeit der Iterationen oder Verteilung der Fitnesswerte. Allerdings könnten auch noch weitere Diagramme bzw. Informationen hinzugefügt werden, z.B. Anzahl der Fahrzeuge, Wartezeiten oder Gesamtfahrzeit.

In diesem Zusammenhang wäre es für den Anwender vorteilhaft, wenn er zur Laufzeit des genetischen Algorithmus Änderungen an den Parametern vornehmen könnte. Eine Möglichkeit wäre das dynamische Ändern der Werte für die Wichtigkeit der Fitnessbausteine.

Momentan wird die Applikation noch aus der Entwicklungsumgebung Eclipse gestartet und die Änderung der Parameter muss direkt im Sourcecode erfolgen. Die Änderung der Parameter erfolgt zwar an einer zentralen Stelle, jedoch muss man Kenntnis über den Aufbau des Programms besitzen. Hier wäre es wünschenswert, wenn die Parameter von außen in die Applikation gegeben werden können. Bestenfalls sollte dies mit einer grafischen Eingabemaske realisiert werden.

Außerdem sind Verbesserungen an der Softwarearchitektur denkbar, beispielsweise könnte noch die Dependency-Injection eingeführt werden, um Abhängigkeiten zwischen Objekten zentral verwalten zu können.

Es wurden bis jetzt nur für einige Module Unittests geschrieben. Es wäre wichtig, dass fehlende Unittests noch ergänzt werden, um nicht offensichtliche Fehler zu finden und im Anschluss zu beheben.

Eine weitere Überlegung wäre, die Werte (Länge eines Plans, Anzahl Fahrzeuge, Fitnesswert) jeder Iteration in eine Datei auszugeben, um im Anschluss detaillierte Auswertungen durchführen zu können.

Momentan beinhaltet die Applikation noch nicht die Umsetzung der Synchronisation zwischen zwei Aufträgen mit verschiedenen Produkten. Diese Funktionalität müsste in der weiteren Entwicklung noch hinzugefügt werden.
