\section{Softwarearchitektur}

\subsection{Systemdesign}
Das Systemdesign setzt sich im wesentlichen aus den folgenden vier Modulen zusammen:
\begin{enumerate}
 \item Konfigurationsgenerierung
 \item Konstruktionsverfahren
 \item Genetisches Optimierungsverfahren
 \item Ergebnisvisualisierung
\end{enumerate}
Desweiteren sind zwei Datenmodelle definiert:
\begin{enumerate}
 \item Konfiguration-Datenmodell
 \item Ergebnis-Datenmodell
\end{enumerate}

\subsection{Konfigurations-Datenmodell}
Das Konfiguration-Datenmodell beschreibt die Konfiguration, die vom Anwender zur Verfügung gestellt wird. Sie umfasst Stationen, Fahrzeuge, Produkte, Aufträge und Zeitfenster (siehe \Fref{fig:KonfigurationDatenmodell}).

\subsection{Ergebnis-Datenmodell}
\myfigure[width=0.95\textwidth]{ErgebnisDatenmodell}{../src/main/java/de/hsbremen/kss/model/Model.png}{Repräsentation eines Plans mit Touren und Aktionen}

\subsection{Konfigurationsgenerierung}

\subsection{Genetischer Algorithmus}
\myfigure[width=0.95\textwidth]{KonfigurationDatenmodell}{../src/main/java/de/hsbremen/kss/genetic/GeneticAlgorithm.png}{Klassendiagramm vom genetischen Algorithmus}