\section{Fazit}
Das Ziel dieses Projektes, die Implementierung eines genetischen Algorithmus zur Optimierung in der Fahrzeugeinsatzplanung, wurde erreicht. Hierbei wurden neue Erfahrungen im Bereich der Fahrzeugeinsatzplanung gesammelt. Die dort vorhandenen Problemstellungen sowie möglichen Lösungswege wurden kennengelernt. Zur Umsetzung des genetischen Algorithmus konnte auf bereits vorhandene Kenntnisse der Softwareentwicklung zurückgegriffen werden. Dadurch bot dieses Projekt eine gute Kombination zwischen neuem und bereits vorhandenem Wissen.

Der genetische Algorithmus ist ein sehr interessanter Ansatz für die Lösung von Optimierungsproblemen. Die Grundidee des genetischen Algorithmus lässt sich relativ einfach umsetzen und liefert gute Ergebnisse. Ein weiterer interessanter Aspekt des genetischen Algorithmus ist seine Vielfältigkeit. Er lässt sich für eine Vielzahl von Optimierungsproblemen einsetzen und könnte daher immer wieder Verwendung in den verschiedensten Gebieten finden.

Ein weiterer wichtiger Punkt bei der Durchführung dieses Projektes war die Komponente Zufall. Es ist erstaunlich, wie viel beim genetischen Algorithmus dem Zufall überlassen wird und trotzdem gute Ergebnisse erzielt werden. 

Während dieses Projektes wurde auch gelernt, wie die Ergebnisse bewertet werden können. Gerade in einem komplexen Gebiet wie der Fahrzeugeinsatzplanung ist es unheimlich schwierig die gefundenen Lösungen einschätzen zu können. Hierbei sind Visualisierungen unheimlich wichtig. Die Verteilung der Touren gibt Auskunft über die Qualität eines Plans, z.B. hat ein guter Plan kaum Überschneidungen und Schleifen. Auch der genetische Algorithmus kann über Visualisierungen bewertet werden, da er einen charakteristische Graph vorweist und ein enger Zusammenhang zwischen dem besten Wert und dem Durchschnittswert einer Population besteht (Selektionsdruck).
