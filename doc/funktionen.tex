% Stellt ein Bild mit Label und Caption dar.
%
% Parameter:
% #1: Größe (optional)
% #2: Label
% #3: Dateiname
% 4#: Caption
\newcommand{\myfigure}[4][width=0.95\textwidth]{
  \begin{figure}[ht!]
    \begin{center}
      \includegraphics[#1]{#3}
      \caption{#4}
      \label{fig:#2}
    \end{center}
  \end{figure}
}

% Stellt zwei Bilder nebeneinander dar.
% Die gesamte Figure kann mit fig:#1 referenziert werden.
% Die linke Figure kann mit fig:#1_l und die rechte mit fig:#1_r referenziert werden.
%
% Parameter:
% #1: Gesamtes Label
% #2: Gesamte Caption
% #3: Dateiname links
% #4: Größe links
% #5: Caption links
% #6: Dateiname rechts
% #7: Größe rechts
% #8: Caption rechts
\newcommand{\twoFigures}[8]{
  \begin{figure}[ht!]
    \begin{center}
      \subfigure[#5]{
	\label{fig:#1_l}
	\includegraphics[#4]{#3} 
      }
      \hspace{0.5cm}
      \subfigure[#8]{
	\label{fig:#1_r}
	\includegraphics[#7]{#6} 
      }
      \caption{#2}
      \label{fig:#1}
    \end{center}
  \end{figure}
}


% Gibt eine Einheit einheitlich aus.
%
% Parameter:
% #1: Einheit die einheitlich ausgegeben werden soll.
\newcommand{\einheit}[1]{
  \textit{[#1]}
}
 
\newcommand{\processlist}[3][\relax]{% 
  \def\listfinish{#1}% 
  \long\def\listact{#2}% 
  \processnext#3\listfinish} 
\newcommand{\processnext}[1]{% 
  \ifx\listfinish#1\empty\else\listact{#1}\expandafter\processnext\fi} 
  
\newcommand{\TODO}[1]{\textbf{\textit{TODO(#1)}}}
