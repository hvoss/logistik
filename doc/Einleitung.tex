\section{Einleitung}
\subsection{Einführung}
Ziel dieses Projektes ist es, ein Programm zu entwickeln, mit dem eine Fahrzeugeinsatzplanung unter Berücksichtigung bestimmter Restriktionen durchgeführt werden kann. 

% TODO Beschreiben was die Fahrzeugeinsatzplanung ist?

\subsection{Anforderungen}
\label{sec:Anforderungen}
Dem Programm wird eine Konfiguration übergeben, in der folgende Parameter beschrieben sind:
\begin{itemize}
 \item Fahrzeuge
 \begin{itemize}
  \item Ein Fahrzeug kann einen Produkt transportieren
  \item Durchschnittliche Geschwindigkeit
  \item Kapazität (maximale Anzahl der Produkte)
  \item Zeitfenster (Arbeitsbeginn und -ende)
  \item Start- und End-Depot (Station)
 \end{itemize}
 \item Produkte
 \begin{itemize}
  \item Name des Produkts
 \end{itemize}
 \item Station
 \begin{itemize}
  \item Name der Station
  \item X-Y-Koordinaten
 \end{itemize}
 \item Auftrag
 \begin{itemize}
  \item Name des Auftrages
  \item Zu transportierendes Produkt
  \item Anzahl der Produkte
  \item Beliebige Auf- und Ablade-Station
  \item Zeitfenster bei Auf- und Ablade-Station
 \end{itemize}
\end{itemize}
Die Entfernung zwischen den Stationen kann über die Koordinaten berechnet werden. 

\subsection{Restriktionen}
\label{sec:Restriktionen}
Folgende Restriktionen sind bei der Planung zu berücksichtigen:
\begin{itemize}
 \item Alle Aufträge müssen erledigt werden
 \item Ein Fahrzeug darf nicht überladen werden
 \item Ein Fahrzeug startet und endet im Depot 
 \item Ein Fahrzeug darf nur innerhalb seines Zeitfensters fahren
 \item Ein Produkt muss zuerst aufgeladen werden, bevor es abgeladen werden kann
 \item Das Auf- und Abladen muss innerhalb des Zeitfensters erfolgen
\end{itemize}
