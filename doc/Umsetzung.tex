\section{Umsetzung}

\subsection{Verwendete Bibliotheken}
Im Rahmen der Entwicklung wurden diverse OpenSource - Bibliotheken eingeführt, um die Entwicklung zu beschleunigen und die Qualität der Software zu steigern. Im folgenden werden die verwendeten Bibliotheken beschrieben:
\begin{description}
 \item[Apache Commons Math] (Version 3.2, Apache License 2.0) \cite{apache:CommonsMath} Stellt mathematische Funktionen zur Verfügung. Wird unteranderem für die Berechnung der Entfernungen verwendet.
 \item[Apache Commons Lang] (Version 3.3.1, Apache License 2.0) \cite{apache:CommonsLang} % TODO wird es wirklich verwendet?
 \item[Apache Commons Collections] (Version 4.0, Apache License 2.0) \cite{apache:CommonsCollection} Erweitert das Java-Collec\-tions-Framework. 
 \item[Joda-Time] (Version 2.3, Apache License 2.0) \cite{joda:jodatime} % TODO wird es wirklich verwendet?
 \item[JUnit] (Version 4.11, Eclipse Public License) \cite{junit:junit} JUnit ist die Standard-Bibliothek für Unit-Tests unter Java. 
 \item[Hamcrest] (Version 1.3, BSD 3-Clause) \cite{hamcrest:hamcrest} Mit Hamcrest ist es möglich bei Tests "`sprechendere"' Ausdrücke zu formulieren.
 \item[EasyMock] (Version 3.2, Apache License 2.0) \cite{easymock:easymock} EasyMock ermöglicht ein einfaches Erstellen von Mock-Objekten für den Unit-Test.
 \item[Simple Logging Facade for Java (SL4J)] (Version 1.7.6, MIT license) \cite{qos:slfj} SLF4J ist eine Log\-ging-Schnittstelle und wird zur Ausgabe auf der Konsole verwendet.
 \item[Logback] (Version 1.1.1, Eclipse Public License v1.0 und LGPL 2.1) \cite{qos:logback} Wird als Implementierung für SLF4J eingesetzt.
 \item[Google Guave] (Version 17.0, Apache License 2.0) \cite{google:guave} Guave ist eine Sammlung von Softwarebibliotheken. Es wird ausschließlich der EventBus für die Kommunikation zwischen den Softwarekomponenten verwendet.
 \item[JFreeChart] (Version 1.0.17, LPGL) \cite{ObjectRefineryLimited:JFreeChart} JFreeChart wird zur Darstellung der Graphen verwendet.
\end{description}

% TODO Entwicklungsumgebung beschreiben?

\subsection{Mutation}
\myfigure[width=0.95\textwidth]{Mutation}{../src/main/java/de/hsbremen/kss/genetic/mutation/Mutation.png}{Implementierte Mutationsalgorithmen}

\subsection{Grafische Oberfläche}
\label{sec:GrafischeOberflaeche}

\myfigure[width=0.95\textwidth]{FitnessDistribution}{FitnessDistribution.png}{Verteilung des Fitnesswertes} % TODO Bildunterschirft
\myfigure[width=0.95\textwidth]{FitnessLengthGraph}{FitnessLengthGraph.png}{} % TODO Bildunterschirft
\myfigure[width=0.95\textwidth]{Map}{Map.png}{} % TODO Bildunterschirft


